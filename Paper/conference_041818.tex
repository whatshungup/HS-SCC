\documentclass[conference]{IEEEtran}
\IEEEoverridecommandlockouts
% The preceding line is only needed to identify funding in the first footnote. If that is unneeded, please comment it out.
\usepackage{cite}
\usepackage{amsmath,amssymb,amsfonts}
\usepackage[utf8]{inputenc}
\usepackage{algorithmic}
\usepackage{graphicx}
\usepackage{textcomp}
\usepackage{xcolor}
%\def\BibTeX{{\rm B\kern-.05em{\sc i\kern-.025em b}\kern-.08em
 %   T\kern-.1667em\lower.7ex\hbox{E}\kern-.125emX}}
\begin{document}

\title{Mehrwert rollenbasierter Umsetzung von kollaborativen Lernumgebungen\\}


\author{\IEEEauthorblockN{Hung Tran Duc}
\IEEEauthorblockA{\textit{Technische Universität Dresden} \\
Dresden, Deutschland \\
hung.tduc@yahoo.com}
}



\maketitle

\begin{abstract} Rollenbasierte Sprachen werden seit Jahrzehnten als Alternative zu objektorientierten Ansätzen untersucht. Die Natur der Rolle erlaubt es Obbjekten, sich dynamisch and verschiedene Anforderungen anzupassen. Mit zunehmend komplexer und offener Software, besteht eine wachsende Nachfrage nach adaptiven Systemen. Kollaborative Lernumgebungen fördern je nach Anwendung die Sozialkompetenz, Eigeninitiative oder Konzentrationsfähigkeit der Lernenden. In dieser Arbeit soll beleuchtet werden, welche Vorzüge eine Lernumgebung auf der Grundlage von Rollen aufweist.
\end{abstract}

\begin{IEEEkeywords}
kollaborative Lernumgebung, e-Learning, rollenbasiert
\end{IEEEkeywords}

\section{Einleitung}
Motivation Rollen:
- RML seit Jahrzehnten untersucht -> Potenzial (Bachman) \cite{family}
- berücksichtigen Kontextabhängiges und kollaboratives Verhalten von Objekten
- Breite an Bereichen: Datenmodelle, Konzeptmodelle, Programmiersprache (Loebe 2005 ; Guarino und welty 2009, Halpin 2005; Hennicker 2015)
- Aktuelle Software: zunehmend komplex und Kontextabhängig(offen,verteilt) -> Nachfrage nach OOP-Alternativen
- Modellsprachen beschreiben Struktur aber nicht dynamisches verhalten. (Reenskaug 2009)
- Verhalten kann unabhängig vom Objekt sein
- ermöglicht adaptives verhalten

Motivation  Collab E-Learning:
- Eigenverantwortung und Initiative
- gemeinsamer austausch
- höhere motivation durch gruppengefühl, 
- ausprägung sozialkompetenz
- voranschreitende digitalisierung (VON ALLEM) 

Motivation Kombination:
- evtl wechselnde Rollen 
- Szenarien? 

Ziel:
- Diskussion/Ausblick

Aufbau:
- Herausforderungen OOM
- Herausforderungen Kollaborative e-learning umgebungen
- Vorteile/stärken rollenbasierter Ansatz
- entstehende Herausforderungen/probleme
- Diskussion und Fazit

Zum ersten Mal definiert in 1977 von Charles w.bachmann


\section{Herausforderungen State of the art}
\subsection{oom}
- beschreiben gut die struktur, nicht das dynamische verhalten
- supplier and customer / Multiple classes / State-dependance
- 
\subsection{Herausforderungen Kollab e-Learning}
- experiences: 
	- Orga und aufgabenverteilung
	- Koordinator erwünscht
	- mehr gruppengefühl erwünscht
	- ungleichmäßige beteiligung
- Soziale interaktion nicht immer gewährleistet
	- soziale interaktion oft auf sehr unpersönlicher ebene
- intelligence assistant(asynchronous communication):
	- Notwendigkeit eines "Lehrers" der bei Problemen in Gruppen eingreift
		- Probleme: Passive Studenten, atypische Teilnahme, gruppen schaffen gar keine tasks, content von Material von niemandem gelesen, keine diskussion begonnen.
- wechsel zwischen lehrer und lernender
- Kontexte? Tasks?  
	- Lehrer und Lernender
	- Koordinator einer Gruppe
	- Gruppe suchen
	- Überwacher / Ansprechpartner
	- Kursleiter auf Abruf/Bereitschaft
	- reviewer/reviewee


\section{role concept and features}
- 26 Features von Rollen vorstellen (von steimann und kühn)
- Erweiterung zu OOM: Objekte wechseln Rollen zur Laufzeit
- wenige State of the Art-ansätze vorstellen


\section{Vorteile rollenbassierter Ansätze/Anwendung}
- wo könnten die 26 features helfen? 
- trennung von aufgaben/problemen
- Trennung von dynamischem und statischem
- dynamische veränderungen des systems -> ANPASSBARKEIT
- langlebigkeit
- stärker je mehr contextwechsel
	-je mehr kontexte, desto mehr kontextwechsel
		-je mehr tasks desto mehr kontextwechsel
\section{Probleme von Rollen}
- wenig support
- uneinigkeit über den begriff 
- produktion von vielen daten
- langsaaam \cite{bachman}


\section{Diskussion und Fazit}

\bibliography{mybib}{}
\bibliographystyle{plain}




\end{document}
