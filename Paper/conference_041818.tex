\documentclass[conference]{IEEEtran}
\IEEEoverridecommandlockouts
% The preceding line is only needed to identify funding in the first footnote. If that is unneeded, please comment it out.
\usepackage{cite}
\usepackage{amsmath,amssymb,amsfonts}
\usepackage[utf8]{inputenc}
\usepackage{algorithmic}
\usepackage{graphicx}
\usepackage{textcomp}
\usepackage{xcolor}
\def\BibTeX{{\rm B\kern-.05em{\sc i\kern-.025em b}\kern-.08em
    T\kern-.1667em\lower.7ex\hbox{E}\kern-.125emX}}
\begin{document}

\title{Mehrwert Rollenbasierter Umsetzung in kollaborativen Lernumgebungen\\}


\author{\IEEEauthorblockN{Hung Tran Duc}
\IEEEauthorblockA{\textit{Technische Universität Dresden} \\
Dresden, Deutschland \\
hung.tduc@yahoo.com}
}



\maketitle

\begin{abstract}
-Motivation Rollen 
-Motivation kollaborative Umgebungen
-Was mache ich 

Rollenbasierte Sprachen werden seit Jahrzehnten als Alternative zu objektorientierten Ansätzen untersucht. Ihre Natur erlaubt es Objekten, sich verschiedenen Begebenheiten und Umständen anzupassen. Mit zunehmend komplexer und offener Software, ist eine Nachfrage nach adaptiven Lösungen entstanden. 
\end{abstract}

\begin{IEEEkeywords}
kollaborative Lernumgebung, e-Learning, rollenbasiert
\end{IEEEkeywords}

\section{Einleitung}
Motivation Rollen:
- RML seit Jahrzehnten untersucht -> Potenzial
- berücksichtigen Kontextabhängiges und kollaboratives Verhalten von Objekten
- Breite an Bereichen: Datenmodelle, Konzeptmodelle, Programmiersprache,
- Aktuelle Software: zunehmend komplex und Kontextabhängig(offen,verteilt) -> Nachfrage nach OOP-Alternativen
- Modellsprachen beschreiben Struktur aber nicht dynamisches verhalten.
- Verhalten kann unabhängig vom Objekt sein
- ermöglicht adaptives verhalten

Motivation  Collab E-Learning:
- Eigenverantwortung und Initiative
- gemeinsamer austausch
- höhere motivation durch gruppengefühl, 
- ausprägung sozialkompetenz
- voranschreitende digitalisierung (VON ALLEM) 

Motivation Kombination:
- evtl wechselnde Rollen
- Szenarien? 

Ziel:
- Diskussion/Ausblick

Aufbau:
- Herausforderungen OOM
- Herausforderungen Kollaborative e-learning umgebungen
- Vorteile/stärken rollenbasierter Ansatz
- entstehende Herausforderungen/probleme
- Diskussion und Fazit


\section{Herausforderungen State of the art}
\subsection{oom}
- beschreiben gut die struktur, nicht das dynamische verhalten
- supplier and customer / Multiple classes / State-dependance
- 
\subsection{Herausforderungen Kollab e-Learning}
- experiences: 
	- Orga und aufgabenverteilung
	- Koordinator erwünscht
	- mehr gruppengefühl erwünscht
	- ungleichmäßige beteiligung
- Soziale interaktion nicht immer gewährleistet
	- soziale interaktion oft auf sehr unpersönlicher ebene
- intelligence assistant(asynchronous communication):
	- Notwendigkeit eines "Lehrers" der bei Problemen in Gruppen eingreift
		- Probleme: Passive Studenten, atypische Teilnahme, gruppen schaffen gar keine tasks, content von Material von niemandem gelesen, keine diskussion begonnen.
- wechsel zwischen lehrer und lernender
- Kontexte? Tasks?  
	- Lehrer und Lernender
	- Koordinator einer Gruppe
	- Gruppe suchen
	- Überwacher / Ansprechpartner
	- Kursleiter auf Abruf/Bereitschaft
	- reviewer/reviewee


\section{role concept and features}
- 26 Features von Rollen vorstellen (von steimann und kühn)
- Erweiterung zu OOM: Objekte wechseln Rollen zur Laufzeit
- wenige State of the Art-ansätze vorstellen


\section{Vorteile rollenbassierter Ansätze/Anwendung}
- wo könnten die 26 features helfen? 
- trennung von aufgaben/problemen
- Trennung von dynamischem und statischem
- dynamische veränderungen des systems -> ANPASSBARKEIT
- langlebigkeit
- stärker je mehr contextwechsel
	-je mehr kontexte, desto mehr kontextwechsel
		-je mehr tasks desto mehr kontextwechsel

\section{Probleme von Rollen}
- wenig support
- uneinigkeit über den begriff 
- produktion von vielen daten
- langsam


\section{Diskussion und Fazit}




\begin{thebibliography}{00}
\bibitem{b1} G. Eason, B. Noble, and I. N. Sneddon, ``On certain integrals of Lipschitz-Hankel type involving products of Bessel functions,'' Phil. Trans. Roy. Soc. London, vol. A247, pp. 529--551, April 1955.
\bibitem{b2} J. Clerk Maxwell, A Treatise on Electricity and Magnetism, 3rd ed., vol. 2. Oxford: Clarendon, 1892, pp.68--73.
\bibitem{b3} I. S. Jacobs and C. P. Bean, ``Fine particles, thin films and exchange anisotropy,'' in Magnetism, vol. III, G. T. Rado and H. Suhl, Eds. New York: Academic, 1963, pp. 271--350.
\bibitem{b4} K. Elissa, ``Title of paper if known,'' unpublished.
\bibitem{b5} R. Nicole, ``Title of paper with only first word capitalized,'' J. Name Stand. Abbrev., in press.
\bibitem{b6} Y. Yorozu, M. Hirano, K. Oka, and Y. Tagawa, ``Electron spectroscopy studies on magneto-optical media and plastic substrate interface,'' IEEE Transl. J. Magn. Japan, vol. 2, pp. 740--741, August 1987 [Digests 9th Annual Conf. Magnetics Japan, p. 301, 1982].
\bibitem{b7} M. Young, The Technical Writer's Handbook. Mill Valley, CA: University Science, 1989.
\end{thebibliography}
\vspace{12pt}


\end{document}
