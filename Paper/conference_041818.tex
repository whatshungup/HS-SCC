\documentclass[conference]{IEEEtran}
\IEEEoverridecommandlockouts
% The preceding line is only needed to identify funding in the first footnote. If that is unneeded, please comment it out.
\usepackage{cite}
\usepackage{amsmath,amssymb,amsfonts}
\usepackage[utf8]{inputenc}
\usepackage{algorithmic}
\usepackage{graphicx}
\usepackage{textcomp}
\usepackage{xcolor}
%\def\BibTeX{{\rm B\kern-.05em{\sc i\kern-.025em b}\kern-.08em
 %   T\kern-.1667em\lower.7ex\hbox{E}\kern-.125emX}}
\begin{document}

\title{Mehrwert der rollenbasierten Umsetzung von kollaborativen Lernumgebungen\\}


\author{\IEEEauthorblockN{Hung Tran Duc}
\IEEEauthorblockA{\textit{Technische Universität Dresden} \\
Dresden, Deutschland \\
hung.tduc@yahoo.com}
}



\maketitle

\begin{abstract} Rollenbasierte Sprachen werden seit Jahrzehnten als Alternative zu objektorientierten Ansätzen untersucht. Die Natur der Rolle erlaubt es Obbjekten, sich dynamisch and verschiedene Anforderungen anzupassen. Mit zunehmend komplexer und offener Software, besteht eine wachsende Nachfrage nach adaptiven Systemen. Kollaborative Lernumgebungen fördern je nach Anwendung die Sozialkompetenz, Eigeninitiative oder Konzentrationsfähigkeit der Lernenden. In dieser Arbeit soll beleuchtet werden, welche Vorzüge eine Lernumgebung auf der Grundlage von Rollen aufweist.
\end{abstract}

\begin{IEEEkeywords}
kollaborative Lernumgebung, e-Learning, rollenbasiert
\end{IEEEkeywords}

\section{Einleitung}
% Motivation Rollen: - RML seit Jahrzehnten untersucht -> Potenzial (Bachman) \cite{bachman} - Breite an Bereichen: Datenmodelle, Konzeptmodelle, Programmiersprache (Loebe 2005 ; Guarino und welty 2009, Halpin 2005; Hennicker 2015) - Aktuelle Software: zunehmend komplex und Kontextabhängig(offen,verteilt) -> Nachfrage nach OOP-Alternativen
% Motivation, warum ich von rollen erzähle: alternative zu oom, lang und breit untersucht, potenzial, aktuelle Nachfrage
%oom ist dominierend, trotzdem lohnt es sich alternativen zu überlegen für innovative systeme, welche möglichkeiten aufzeigen könnten, welche mit OO noch nicht erkenntlich waren
Objektorientierte Programmierung ist das verbreiteste, am häufigsten angewendete und ein weitläufig akzeptiertes Programmierparadigma. Dennoch kann es sich lohnen, Alternativen zu untersuchen. Sie könnten zu innovativen Konzepten führen, welche Mittel und Wege aufzeigen, die mit Objektorientierung nicht erkenntlich waren. Beispielalternativen sind funktionsorientierte Programmierung, welche sich auf die Modellierung von Prozessen fokusierte, oder aspektorientierte Programmierung. Letztere bietet die Möglichkeit zentral Verhalten, das mehrere unabhängige Klassen annehmen müssen, zu definieren. \\ Rollenbasierte Programmierung stellt eine weitere Alternative zur Objektorientierung dar. Zum ersten Mal wurden Rollen 1977 von Bachman et al. charakterisiert. Sie beschreiben Rollen als festgelegtes Verhalten, welches von Objekten verschiedener Klassen angenommen werden kann \cite{bachman}. Seitdem wurde der Begriff der Rolle stets wieder aufgegriffen und in verschiedenen Bereichen thematisiert. Diese Bereiche umfassen Ontologien\cite{loebe2005abstract}\cite{guarino2009overview}, Datenmodellierung \cite{halpin2005orm}, Konzeptmodellierung \cite{hennicker2015model} und Programmiersprachen \cite{ubayashi2000roleep}. In \cite{steimann2000represantation} hat Steimann aktuelle Untersuchungen zu Rollen zusammengefasst und bewertete den Einfluss der Rolle auf die moderne Datenmodellierung als gering. Diese Beobachtung und die Menge an Untersuchungen, sowohl vor als auch nach Steimann, weisen auf ein großes aber ungenutztes Potenzial der Rolle als Programmierparadigma hin. Eine Rolle beschreibt Attribute und Verhalten von Objekten, die in einem bestimmten Kontext miteinander kollaborieren. Ein Objekt kann also je nach Bedarf eine neue Rolle und somit auch neues Verhalten annehmen. Diese Anpassungsfähigkeit kann behilflich sein, da moderne Softwaresysteme zunehmend komplexer und offener werden \cite{murer2008managed}. \\
%Motivation  Collab E-Learning:- Eigenverantwortung und Initiative- gemeinsamer austausch- höhere motivation durch gruppengefühl, - ausprägung sozialkompetenz- voranschreitende digitalisierung (VON ALLEM) 
% Motivation für kollaborative Lernumgebungen: was sind sie/neu, wachsend -> Potenzial, verbreitung(Arten,beispiele), vorteile: gruppengefühl, aufmerksamkeit, eigeninitiative, globalität, 
In den meisten Hochschulen werden mittlerweile Lernplattformen verwendet. Neben der Bereitstellung von Lehrmaterialien, dienen sie auch zur Organisation von Lernvorgängen. Als Lernumgebung erleichtern sie den Lernenden die Kommunikation untereinander oder mit Lehrern. Der Lernprozess kann somit zu beliebiger Zeit an einem beliebigen Ort mit einem mobilen Endgerät durchgeführt werden. Audience Response Systeme (ARS) werden während Lehrveranstaltungen eingesetzt. Sie ermöglichen es, von allen Zuhörern gleichzeitig Feedback bzw. Input einzuholen.Browserbasiert oder als mobile App verfügbar , benötigen die Systeme keine spezielle Hardware. Primärer Anwendungsfall ist das Stellen von Fachfragen um den Vorbereitungsgrad oder den Lernstand des Publikums zu ermitteln. Es konnte beobachtet, dass das ARS zu einem deutlich höherem Engagement in Vorlesungen führt. Zusätzlich konnten bei Studenten eine kontinuierliche Aufmerksamkeit und höhere Lernleistung festgestellt werden \cite{digitaleHochschule}. Eine andere Anwendung stellen kollaborative Lernumgebungen dar. Sie haben ein einfaches Grundprinzip: Lernende werden in Gruppen geteilt um gemeinsam Aufgaben zu lösen. Sie teilen die Aufgaben untereinander auf und tauschen ihre Kenntnisse aus. Auf diese Weise werden Sozialkompetenz und Organisationsfähigkeit gefördert. Auch die Motivation eines Studierenden, kann durch das Gefühl der Gruppenzugehörigkeit gesteigert werden. Der Fokus auf Kollaboration, sowie die verschiedenen Anwendungsfälle legen nahe, dass Lernumgebungen aus einer rollenbasierten Umsetzung einen Vorteil ziehen könnten. \\ In dieser Arbeit wird diskutiert, welche Vorzüge eine rollenbasierte Lernumgebung gegenüber einer klassisch objektorientierten Umsetzung haben könnte. Dazu werden in Kapitel 2 zuerst die Probleme und Einschränkungen der objektorientierten Programmierung und Modellierung zusammengefasst. Nachfolgend wird auf die Herausforderungen von kollaborativen Lernumgebungen eingegangen. Kapitel 3 thematisiert die möglichen Vorteile von rollenbasierten Modellierungs- und Programmiersprachen, welche einer Kollaborativen Lernumgebung von Nutzen sein könnten. Die eventuell entstehenden Herausforderungen und Probleme rollenbasierter Ansätze werden in Kapitel 4 erläutert.



\section{Herausforderungen State of the art}
\subsection{oom}
- beschreiben gut die struktur, nicht das dynamische verhalten (Reenskaug 2009)
- supplier and customer / Multiple classes / State-dependance
- 
\subsection{Herausforderungen Kollab e-Learning}
- experiences: 
	- Orga und aufgabenverteilung
	- Koordinator erwünscht
	- mehr gruppengefühl erwünscht
	- ungleichmäßige beteiligung
- Soziale interaktion nicht immer gewährleistet
	- soziale interaktion oft auf sehr unpersönlicher ebene
- intelligence assistant(asynchronous communication):
	- Notwendigkeit eines "Lehrers" der bei Problemen in Gruppen eingreift
		- Probleme: Passive Studenten, atypische Teilnahme, gruppen schaffen gar keine tasks, content von Material von niemandem gelesen, keine diskussion begonnen.
- wechsel zwischen lehrer und lernender
- Kontexte? Tasks?  
	- Lehrer und Lernender
	- Koordinator einer Gruppe
	- Gruppe suchen
	- Überwacher / Ansprechpartner
	- Kursleiter auf Abruf/Bereitschaft
	- reviewer/reviewee


\section{role concept and features}
- 26 Features von Rollen vorstellen (von steimann und kühn)
- Erweiterung zu OOM: Objekte wechseln Rollen zur Laufzeit
- wenige State of the Art-ansätze vorstellen
- RoSI vorstellen
	- CROM (Compartment Role Objekt Model) vorstellen
		-Tool: FRAMEDA

\section{Vorteile rollenbassierter Ansätze/Anwendung}
- wo könnten die 26 features helfen? 
- trennung von aufgaben/problemen
- Trennung von dynamischem und statischem
- dynamische veränderungen des systems -> ANPASSBARKEIT ZUR LAUFZEIT
	- gruppen in lernumgebungen nicht vorhersehbar
- Langlebigkeit
- berücksichtigen Kontextabhängiges und kollaboratives Verhalten von Objekten \cite{family}
- Verhalten kann unabhängig vom Objekt sein - ermöglicht adaptives verhalten\\
- stärker je mehr contextwechsel
	-je mehr kontexte, desto mehr kontextwechsel
		-je mehr tasks desto mehr kontextwechsel
\section{Probleme von Rollen}
- wenig support
- uneinigkeit über den begriff 
- produktion von vielen daten
- langsam \cite{bachman}


\section{Diskussion und Fazit}

\bibliography{mybib}{}
\bibliographystyle{plain}




\end{document}
