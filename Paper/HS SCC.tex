
\documentclass[prodmode,acmtap]{acmlarge}

% Metadata Information
\acmVolume{2}
\acmNumber{3}
\acmArticle{1}
\articleSeq{1}
\acmYear{2010}
\acmMonth{5}

% Package to generate and customize Algorithm as per ACM style
\usepackage[ruled]{algorithm2e}
\SetAlFnt{\algofont}
\SetAlCapFnt{\algofont}
\SetAlCapNameFnt{\algofont}
\SetAlCapHSkip{0pt}
\IncMargin{-\parindent}
\renewcommand{\algorithmcfname}{ALGORITHM}

% Page heads
\markboth{D. Pineo, C. Ware and S. Fogarty}{Neural Modeling of Flow Rendering Effectiveness}

% Title portion
\title{Mehrwert rollenbasierter Konzepte in kollaborativen Lernumgebungen}
\author{Hung Tran Duc \affil{Technische Universität Dresden} }


\begin{abstract}

\end{abstract}

\category{H.5.2}{Information Interfaces and Presentation}{User
Interfaces}[Evaluation/\break methodology]
\category{H.1.2}{Models and Principles}{User/Machine Systems}[Human Information Processing]
\category{I.5.1}{Pattern\break Recognition}{Models}[Neural Nets]

\terms{}
\keywords{}

\acmformat{Daniel Pineo, Colin Ware, and Sean Fogarty. 2010. Neural Modeling of Flow Rendering Effectiveness.}

\begin{document}

\begin{bottomstuff}
This work is supported by the Widget Corporation Grant \#312-001.\\
Author's address: D. Pineo, Kingsbury Hall, 33 Academic Way, Durham,
N.H. 03824; email: dspineo@comcast.net; Colin Ware, Jere A. Chase
Ocean Engineering Lab, 24 Colovos Road, Durham, NH 03824; email: cware@ccom.unh.edu;
Sean Fogarty, (Current address) NASA Ames Research Center, Moffett Field, California 94035.
\end{bottomstuff}


\maketitle

% Head 1
\section{Einleitung}
Motivation Rollen:
- RML seit Jahrzehnten untersucht
- berücksichtigen Kontextabhängiges und kollaboratives Verhalten von Objekten
- Breite an Bereichen: Datenmodelle, Konzeptmodelle, Programmiersprache,
- Aktuelle Software: zunehmend komplex und Kontextabhängig(offen,verteilt) -> Nachfrage nach OOP-Alternativen
- Modellsprachen beschreiben Struktur aber nicht dynamisches verhalten.
- Verhalten kann unabhängig vom Objekt sein
- ermöglicht adaptives verhalten

Motivation  Collab E-Learning:

Aufbau:

\section{Herausforderungen OOM}
\section{Herausforderungen Kollab e-Learning}
\section{Anwendung rollenbassierter Ansätze}
\section{Nachteile}
\section{Diskussion und Fazit}






\section{Discussion}


% Appendix
\appendix
\section*{APPENDIX}
\setcounter{section}{1}




% Bibliography
\bibliographystyle{ACM-Reference-Format-Journals}
\bibliography{acmlarge-sample-bibfile}
                                % Sample .bib file with references that match those in
                                % the 'Specifications Document (V1.5)' as well containing
                                % 'legacy' bibs and bibs with 'alternate codings'.
                                % Gerry Murray - March 2012

% History dates
\received{February 2009}{July 2009}{October 2009}


\elecappendix


\section{Analysis of Invalid Trials}
\label{invalid}

\subsection{Results}





\subsection{Discussion}



\end{document}
% End of v2-acmlarge-sample.tex (March 2012) - Gerry Murray, ACM
